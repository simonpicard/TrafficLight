%%%%%%%%%%%%%%%%%%%%%%%%%%%%%%%%%%%%%%%%%%%%%%%%%%%%%%%%%%%%%%%%%%%%%%
% LaTeX Template: Beamer arrows
%
% Source: http://www.texample.net/
% Feel free to distribute this template, but please keep the
% referal to TeXample.net.
% Date: Nov 2006
% 
%%%%%%%%%%%%%%%%%%%%%%%%%%%%%%%%%%%%%%%%%%%%%%%%%%%%%%%%%%%%%%%%%%%%%%
% How to use writeLaTeX: 
%
% You edit the source code here on the left, and the preview on the
% right shows you the result within a few seconds.
%
% Bookmark this page and share the URL with your co-authors. They can
% edit at the same time!
%
% You can upload figures, bibliographies, custom classes and
% styles using the files menu.
%
% If you're new to LaTeX, the wikibook is a great place to start:
% http://en.wikibooks.org/wiki/LaTeX
%
%%%%%%%%%%%%%%%%%%%%%%%%%%%%%%%%%%%%%%%%%%%%%%%%%%%%%%%%%%%%%%%%%%%%%%

\documentclass{beamer} %
\usetheme{CambridgeUS}
\usepackage[utf8]{inputenc}
\usefonttheme{professionalfonts}
\usepackage{times}
\usepackage{tikz}
\usepackage{amsmath}
\usepackage{verbatim}
\usetikzlibrary{arrows,shapes}

\author{Author}
\title{Presentation title}

\begin{document}

\title[INFO-F-412 - Traffic Light Control]{\textbf{Traffic Light Control} \\INFO-F-412 -- Formal Verification} % The short title appears at the bottom of every slide, the full title is only on the title page

\author{Jamal \textsc{Ben Azouze}, Marien \textsc{Bourguignon}, Nicolas \textsc{De Groote}, \\Simon \textsc{Picard}, Arnaud \textsc{Rosette}, Gabriel \textsc{Ekanga}}
\institute[ULB] % Your institution as it will appear on the bottom of every slide, may be shorthand to save space
{
Université Libre de Burxelles \\ % Your institution for the title page
\medskip
\textit{Département d'Informatique} % Your email address
}
\date{4 Juin 2015} % Date, can be changed to a custom date

\begin{frame}
\titlepage % Print the title page as the first slide
\end{frame}

\begin{frame}{Sommaire}
\tableofcontents % Throughout your presentation, if you choose to use \section{} and \subsection{} commands, these will automatically be printed on this slide as an overview of your presentation
\end{frame}

\section{Introduction}
\begin{frame}{Introduction}

\end{frame}

\section{Uppaal}
\begin{frame}{Uppaal}

\end{frame}

\section{Acteurs et automates}
\subsection{Acteurs de l'environnement}
\begin{frame}{bla}

\end{frame}

\subsection{Automates de l'environnement}
\begin{frame}{bla}

\end{frame}

\subsection{Contrôleur}
\begin{frame}{Contrôleur}

\end{frame}

\section{Vérification}
\begin{frame}{Propriété ...}

\end{frame}

\section{Conclusion}
\begin{frame}{Conclusion}

\end{frame}
\end{document}